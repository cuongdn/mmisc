\chapter{Introduction}

\section{An Overview about PUF}
Website: http://www.pufhcm.edu.vn/\\
The P\^{o}les Universitaires Fran\c{c}ais (PUF) was created by the intergovernmental agreement of VietNam and French in October 2004. There are two PUF centers in VietNam: PUF HN located in Ha Noi and PUF HCM located in Ho Chi Minh City. PUF has more than 500 students participated in courses in the field of Commerce, Economics, Management and Informatics. Courses in PUF are provided in French, and English and Vietnamese by both the French and Vietnamese professors.  Currently, PUF-HCM cooperates with the University of Toulouse I Capitole, University of Paris VI and the University of Bordeaux 1 providing Bachelor's, Master’s and PHD courses.
%

\section{ELCA introduction}
\subsection{An overview}
Website: https://www.elca.vn/\\
ELCA is Switzerland's largest independent software development company. In 1998 ELCA was one of the first 100\% foreign-owned software companies to open an office in Ho Chi Minh City. The company develops on .NET and JAVA platforms and integrates products like SharePoint and CRM. The quality system is appraised at CMMI maturity level 3. ELCA Vietnam started with six people. Today we are a production facility with more than a hundred employees. During the same period ELCA Switzerland tripled its workforce to over 450 engineers. 
\par
Our customer has used Oracle Forms as their main application platform for the last 20 years. During this time, several hundred applications have been developed. Since 2008, Oracle Forms is no longer supported. The customer now wants to migrate all their applications to a more modern .NET framework. The goals of the project are:
\begin{itemize}
	\item To have a single homogenous and flexible environment;
	\item To obtain clean and transparent processes;
	\item To reduce complexity.
\end{itemize}
\par
The ELCA Company serves as an extension of the development team of the customer and will be tasked with migrating individual applications or packets of applications.
%

\subsection{My supervisor}
I would like to express the deep appreciation to Mr. VO Tran Trong Vu for giving me the opportunity to work with him. His calm and patience support, guidance, advice in terms of the design, methodology, and how to perform the task in an efficient way helped me in my short period of time in ELCA. His prompt responses and thought-provoking arguments have helped me much in completing this result. In addition, I always feel assured and more confident to explore and figure out the best solutions for myself with his support and suggestions. This is really important for me who am so novice in large-scaled projects.
%

\section{An overview of my task}
The duration of the internship is six months, starts from July 1, 2014 to December 30, 2014.
As a member of the development team, I developed several modules for 6 months. During this period of time, I was a developer. My tasks are verifying the specifications in order to understand all the business and technical problems thoroughly. I also participated in developing the assigned modules, debugging the code; writing the unit test for the modules as well as providing necessary support to other team members. 
%

\section{The internship report}
This report involves six chapters. We firstly give a brief overview of our report with its top most basic goal in the first introductory chapter. In the second chapter, we introduce the theme of our project and our company policy in finishing a task. The aim of the project and our customers' requirements will be presented in a condense way in this chapter. In this chapter, we also dive into the challenges we encountered when finishing the project. 
\par
In the fourth chapter, we present the detailed tasks as well as the solutions applied to tackle the problems highlighted in the previous chapter.  To give a deeper understanding about our projects in ELCA, we spend the first section of the chapter to describe the system in more detail. The layer architecture is playing the vital role in our project and a careful understanding in this architecture is necessary for inferring any further related problems and solutions applied in the system. Due to the company's policy, it is forbidden to provide the projects names and its code in a clear way. I write the report in a descriptive way rather than provide a clear example of the whole project.
Finally, in the last chapter, we summarize our report not forgetting to refer to some of the future work for the projects in related fields. 
%

